\documentclass{ltxdoc}

\usepackage{cmap} % для кодировки шрифтов в pdf
\usepackage[T2A]{fontenc}
\usepackage[utf8]{inputenc}
\usepackage[russian]{babel}
\usepackage{tabularx}

\newenvironment{cmddescription}
{
	\tabularx{\textwidth}{@{}lX}
}
{
	\endtabularx
}

\begin{document}
\begin{center}
\Large Класс \textbf{maithesis} для оформления пояснительной записки к дипломной работе в МАИ
\bigskip

\large Владимир Панченко
\bigskip

2013/12/28
\end{center}

\section{Введение}
Данный класс предназначен для оформления пояснительной записки к дипломной работе в Московском Авиационном Институте (МАИ). При разработке были учтены требования ГОСТ 7.32-2001 «Отчёт о научно-исследователькой работе» и ГОСТ 2.105-95 «Общие требования к текстовым документам». Для оформления списка использованных источников используется стиль \textit{utf8gost705u}, основанный на ГОСТ 7.0.5-2008 «Библиографическая ссылка. Общие требования и правила составления». Основные надписи на графический материал оформлены по форме 2 в соответствии с ГОСТ 2.104-2006 «Основные надписи».

Стилевое оформление класса полностью соответствует требованиям кафедры 308 факультета №3 Московского Авиационного Института. Так как класс учитывает требования ГОСТов, то он может использоваться и учащимися других кафедр и факультетов.

\section{Использование}
Команда объявления класса в новом документе должна выглядеть так:
\begin{verbatim}
	\documentclass{maithesis}
\end{verbatim}
Вариант с параметрами:
\begin{verbatim}
	\documentclass[draft,12pt]{maithesis}
\end{verbatim}

Далее указывается информация об авторе и работе (необходима для корректного заполнения реферата и основных надписей):
\begin{verbatim}
\thesistitle{Разработка системы мониторинга ЛА (Integrated System Health Management) на основе методов интеллектуального анализа данных (Data Mining)}
\keywords{контроль состояния, обнаружение аномалий, интеллектуальный анализ данных, Data Mining, кластеризация, Integrated System Health Monitoring, телеметрия}
\author{Панченко В.В.}
\superviser{Пискунов В.А.}
\specadviser{Васильев А.Д.}
\headofsubfaculty{Шаронов А.В.}
\universityname{МАИ}
\specialtynumber{230201}
\thesisyear{2013}
\groupnumber{03-617}
\end{verbatim}

После чего идёт основная часть документа:
\begin{verbatim}
\begin{document}
	\includepdf[pages={1}]{title.pdf} % титульник
	\includepdf[pages={1,2}]{assignment.pdf} % задание
	\setcounter{page}{3} % начать нумерацию страниц с №3
	\include{referat} % реферат
	\tableofcontents % содержание
	
	\include{index} % список сокращений
	\include{intro} % введение
	\include{./special/special} % специальная часть
	\include{./economics/economics} % экономическая часть
	\include{./labourprotection/labourprotection} % охрана труда и окружающей среды
	\include{conclusion} % заключение
	
	%% добавить список использованных источников в содержание
	\addcontentsline{toc}{chapter}{\bibname}
	%% добавить в список использованных источников все элементы, на которые не было
	%% ссылок в тексте, но которые содержатся в bib-файле
	\nocite{*}
	\bibliography{thesis} % вставить список использованных источников
	
	%% Приложения
	\appendix
	\begin{appendices}
		\include{./appendices/gmmscheme}
		...
	\end{appendices}
\end{document}
\end{verbatim}

\section{Настройки}
Класс поддерживает следующие опции.
\subsection{Режим черновика (draft mode)}
Опции \verb|draft|, \verb|final|. Включают/выключают режим черновика (см. документацию по \LaTeXe). Пример использования:
\begin{verbatim}
	\documentclass[draft]{maithesis}
\end{verbatim}
Значение по умолчанию: \verb|final|.
\subsection{Выбор размера шрифта текста}
Опции \verb|12pt|, \verb|14pt|. Устанавливают размер шрифта для обычного текста 12 и 14 пунктов соответственно. Пример использования:
\begin{verbatim}
	\documentclass[draft]{maithesis}
\end{verbatim}
Значение по умолчанию: \verb|14pt|.

\section{Команды класса}
\subsection{Информация о работе}
\begin{cmddescription}
\cmd{\thesistitle}\marg{text} & Задаёт тему дипломной работы. \\
\cmd{\author}\marg{text} & Задаёт фамилию и инициалы автора работы. \\
\cmd{\superviser}\marg{text} & Задаёт фамилию и инициалы дипломного руководителя. \\
\cmd{\specadviser}\marg{text} & Задаёт фамилию и инициалы консультанта по специальной части. \\
\cmd{\headofsubfaculty}\marg{text} & Задаёт фамилию и инициалы заведующего кафедрой. \\
\cmd{\universityname}\marg{text} & Задаёт краткое название университета (по умолчанию~---~«МАИ»). \\
\cmd{\specialtynumber}\marg{number} & Задаёт номер специальности. \\
\cmd{\thesisyear}\marg{year} & Задаёт год написания работы. \\
\cmd{\groupnumber}\marg{text} & Задаёт номер учебной группы. \\
\cmd{\keywords}\marg{text} & Задаёт список ключевых слов по данной работе. Ключевые слова и фразы указываются через запятую.
\end{cmddescription}
\smallskip

Команды с приставкой «the» выводят значение, например, \verb|\thethesistitle| выведет тему дипломной работы.

\subsection{Главы}
\begin{cmddescription}
\cmd{\spchapter}\marg{name} & Специальная глава (например, введение, заключение). Название главы размещается по центру строки, не нумеруется, глава добавляется в содержание. \\
\cmd{\spchapter*}\marg{name} & Вариант команды \cmd{\spchapter}, не добавляющий главу в содержание. \\
\end{cmddescription}

\subsection{Списки}
\begin{cmddescription}
Окружение \texttt{abbreviation} & Список используемых обозначений и сокращений. В использовании аналогично окружению \texttt{description}. \\
\end{cmddescription}
Пример использования:
\begin{verbatim}
	\begin{abbreviation}
		\item[ООП]~---~объектно-ориентированное программирование;
		\item[SVM]~---~Support Vector Machine.
	\end{abbreviation}
\end{verbatim}

\subsection{Таблицы}
\begin{cmddescription}
Новый тип столбца: \texttt{C} & Аналогичен типу \texttt{p}, но центрирует текст ячейки по горизонтали и вертикали. Пример: \verb|\begin{tabular}{|C{3cm}|}| \\
\cmd{\longtablehead}\marg{caption}\marg{header} & Автоматически настраивает заголовки longtable в соответствии с \mbox{ГОСТом}.
\end{cmddescription}

\subsection{Шрифты}
\begin{cmddescription}
\cmd{\GostTypeAFont} & Чертёжный шрифт, тип А. \\
\cmd{\GostTypeBFont} & Чертёжный шрифт, тип B. \\
\end{cmddescription}
\smallskip

По умолчанию для обеих типов используется шрифт \textit{Arial}, как и в ГОСТ 2.104-2006 «Основные надписи».

\subsection{Разное}
\begin{cmddescription}
\cmd{\nohyphenation} & Отключает перенос слов в текущем фрагменте. \\
\end{cmddescription}

\end{document}